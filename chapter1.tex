\chapter{Introduction}
\label{cha:introduction}

Bicycle counter is a device that automatically counts cyclists riding on the road. Roadside counting devices for riding cyclists are installed in many cities, also in Krakow. They are usually part of the broader strategy of the city, wanting to encourage its residents to ride bicycles. The bicycle counter usually works on the principle of induction loops embedded in a nearby bicycle path that detect passing bikes. Sometimes photocells are also used to count cyclists. However, the scope of this work is to create visual bicycle counters, based on techniques for recognizing images from publicly available webcams. These data should be counted over a longer time horizon, and, optimally, be combined with data on cards from automatic, urban cycling measurement points

%---------------------------------------------------------------------------

\section{Goals}
\label{sec:goals}

The goal of my thesis was to create mechanism for counting cyclist on video file, downloaded from publicly available street cameras' view. That in turn could help in evaluation of investments made by city authorities or to plan future projects.


%---------------------------------------------------------------------------

\section{My Contribution}
\label{sec:myContribution}

To achieve mentioned goal I had to conduct research about newest Machine Learning and Computer Vision algorithms and techniques as well as learning how to customize their functioning. As an addition I have not only train AI model to detect and count cyclist, but also used it on more video files to get data and show exemplary visualization and its interpretation.

%---------------------------------------------------------------------------

\section{Order of content}
\label{sec:orderOfContent}
First of all I am going to write about theoretical aspects of my work with \textit{State of the Art} of Computer Vision and Machine Learning in traffic planning and analysis. Then I am going to describe practical side of work I had to do to achieve goals of my thesis. At the end I would like to show results of my work as well as visualization of data my program collects with exemplary interpretation of this data. 














