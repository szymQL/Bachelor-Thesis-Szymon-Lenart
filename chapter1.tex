\chapter{Introduction}
\label{cha:introduction}

Bicycle counter is a device that automatically counts cyclists riding on the road. Roadside counting devices for riding cyclists are installed in many cities, also in Krakow. They are usually part of the city's broader strategy, encouraging its residents to ride bicycles. The bicycle counter usually works on induction loops embedded in a nearby bicycle path that detect passing bikes. Sometimes photocells are also used to count cyclists. However, this work's scope is to create visual bicycle counters, based on recognizing images from publicly available webcams. These data should be counted over a longer time horizon, and, optimally, be combined with data on cards from automatic, urban cycling measurement points.

\section{Goals}
\label{sec:goals}

My thesis's goal was to create a mechanism for counting cyclist on the video file, downloaded from publicly available street cameras' view. That, in turn, could help evaluate investments made by city authorities or plan future projects.

\section{My Contribution}
\label{sec:myContribution}

To achieve the mentioned goal, I had to research the newest Machine Learning and Computer Vision algorithms and techniques and learn how to customize their functioning. As an addition, I have not only trained the AI model to detect and count cyclist but also used it on more video files to get data and show its interpretation.

\section{Order of Content}
\label{sec:orderOfContent}
First of all, I will write about theoretical aspects of my work with \textit{State of the Art} of Computer Vision and Machine Learning in traffic planning and analysis. Then I will describe the practical side of work I had to do to achieve my thesis goals. In the end, I would like to show the results of my work as well as the visualization of data my program collects with the exemplary interpretation of this data. 














